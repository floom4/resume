\documentclass[11pt,a4paper]{moderncv}

\usepackage{moderntimeline}
\usepackage[english]{babel}
\usepackage[utf8x]{inputenc}
\usepackage[T1]{fontenc}
\usepackage[super]{nth}
%\usepackage{xspace}
%\renewcommand{\FrenchLabelItem}{\textcolor{blue}{$\circ$}}

\moderncvstyle{classic}
\moderncvcolor{orange}
\tlmaxdates{2013}{2016}
\tlwidth{0.8ex}
\tltext{\tiny}
% adjust the page margins
\usepackage[scale=0.90]{geometry}
\setlength{\hintscolumnwidth}{2.5cm}           % if you want to change the width of the column with the dates
\renewcommand{\familydefault}{\sfdefault}
\usepackage{helvet}

% personal data
\firstname{Corentin}
\familyname{Le Bigot}
\title{Étudiant \`a Epita en premi\`ere ann\'ee d'ing\'enieur}
\address{XXX XXXXXXXX XXXXX}{XXXXX XXXXXXX, France}
\mobile{(+33)X-XX-XX-XX-XX}
\email{lebigot.corentin@gmail.com}
\extrainfo{21 ans}
\photo[60pt][0.2pt]{photo.png}


\begin{document}

\makecvtitle

\section{Formation}

\cventry{Depuis Sept 2014 }{École d'ing\'enierie en informatique}{EPITA}{Paris}
{France}{Cycle de 5 ans d'\'etude incluant 2 ann\'ees pr\'eparatoires -
Actuellement en troisi\`eme ann\'ee}

\cventry{Jan - Juin 2016}{Cursus d'ing\'enierie en informatique}{Stellenbosch University}{Stellenbosch}
{Afrique du Sud}{Semestre \`a l'\'etranger en continuit\'e de la scolarit\'e \`a EPITA}

\cventry{Juil 2014 }{Baccalaur\'eat en fili\`ere scientifique, 
Mention Assez Bien}
{Lyc\'ee Louis Payen}{Saint Paul, Île de la R\'eunion, France}{}{}

\section{Expérience}

\subsection{Scolaire}

\cventry{Jan 2017}
{D\'eveloppeur}{Compilateur du langage Tiger en C++ (backend inclus)}
{9 $\times$ 2 semaines}{\'equipe de 3}{}

\cventry{Nov 2016}
{D\'eveloppeur}{Linker dynamique bas\'e sur le ld.so de UNIX en C++ et assembleur x86}
{2 semaines}{}{}

\cventry{Nov 2016}{D\'eveloppeur}
{Shell respectant la norme Posix en C bas\'e sur bash}
{3 semaines}{\'equipe de 3}{}

\cventry{Sept 2015}{D\'eveloppeur}
{logiciel de reconnaissance faciale en C (Bas\'e sur l'algorithme de Viola \& Jones)}
{4 mois}{\'equipe de 4}{}

\cventry{Jan 2015}{Chef de projet/D\'eveloppeur}
{jeu de strat\'egie en temps r\'eel en C\#}
{6 mois}{\'equipe de 3}{}

\subsection{Professionnelle}

\cventry{Sept 2016- Juin 2017}{Assistant enseignant OCaml/C\#}
{EPITA}{Paris, France}{Enseignement du C\# et du OCaml aux \'eleves de premi\`ere ann\'ee \`a EPITA}{}

\cventry{Juil - Ao\^ut 2016}{D\'eveloppeur}
{Sogeti High Tech}{Toulouse, France}
{Stage d'un mois dans un service de R\&D. Synchronisation d'un bras r\'eel \`a un bras mod\'elis\'e en 3D \`a l'aide du bracelet Myo.}{}

\section{Comp\'etences}

\cvitem{Programmation}{C, C++, C\#, OCaml, Python, Java, Shell Scripting, Assembleur x86}{}{}
\cvitem{Autre}{Environnement Unix, \LaTeX, Versioning (Git)}{}{}

\section{Langues}
\cvlanguage{Anglais}{Courant} {Score de 825 au TOEIC + 1 semestre en Afrique du Sud}
\cvlanguage{Espagnol}{Rudiments}{}

\section{Centres d'int\'er\^ets}

\cvitem{Organisation}{\textbf{Membre de l'association GConfs} (l'association GConfs organise des conf\'erences sur les campus de l'EPITA) Gestion de la partie administrative pour l'organisation de conf\'erences.
\href{http://www.gconfs.fr/}{(http://www.gconfs.fr)}}
\cvitem{Sport}{Musculation, footing}
\cvitem{Divers}{Programmation, Lecture, Skateboard}

\end{document}
